\newpage
\section{Anexa}

\subsection{XML}
\subsubsection{Motion body}
Exemplu simplificat de element care reprezintă un corp de mișcare în fișierul .mdef:
\lstset{language=XML}
\begin{lstlisting}
<MotionBody Name="m1">
    <MassAndInertia>
        <TransformationMatrix>
            <Origin>
                <X Unit="in" Value="-7.58042829409448870592e+00"/>
                <Y Unit="in" Value="4.23044650011811071977e+01"/>
                <Z Unit="in" Value="3.44301163838582695575e+01"/>
            </Origin>
        </TransformationMatrix>
    </MassAndInertia>
</MotionBody>

\end{lstlisting}

\subsubsection{Joint}
Exemplu simplificat de element care reprezintă o legătură între două corpuri de mișcare:

\lstset{language=XML}
\begin{lstlisting}
<Joint Name="u_jt_shock_l" Type="Cylindrical">
    <Action>
        <SelectedMotionBody Name="shock_cyl_l"/>
        <TransformationMatrix>
            <Origin>
                <X Unit="in" Value="-4.14178975641839031141e+00"/>
                <Y Unit="in" Value="2.03613060274620565338e+01"/>
                <Z Unit="in" Value="4.14651875259415945152e+01"/>
            </Origin>
        </TransformationMatrix>
    </Action>
    <Base>
        <SelectedMotionBody Name="bell_crank_l"/>
        <TransformationMatrix>
            <Origin>
                <X Unit="in" Value="-4.14178975641839031141e+00"/>
                <Y Unit="in" Value="2.03613060274620565338e+01"/>
                <Z Unit="in" Value="4.14651875259415945152e+01"/>
            </Origin>
        </TransformationMatrix>
        <SnapMotionBodies Value="false"/>
    </Base>
    <DisplayScale Value="1.00000000000000000000e+00"/>
</Joint>
\end{lstlisting}

\subsubsection{Connector}
Exemplu de element Connector de tip arc (spring) dintre doua corpuri de miscare:

\lstset{language=XML}
\begin{lstlisting}
<Spring Name="S001" Type="MotionBody">
    <Action>
        <SelectedMotionBody Name="shock_rod_l"/>
        <TransformationMatrix>
            <Origin>
                <X Unit="in" Value="4.97282173372801139521e+00"/>
                <Y Unit="in" Value="2.22172095074280626648e+01"/>
                <Z Unit="in" Value="3.98239507981948648307e+01"/>
            </Origin>
        </TransformationMatrix>
    </Action>
    <Base>
        <SelectedMotionBody Name="shock_cyl_l"/>
        <TransformationMatrix>
            <Origin>
                <X Unit="in" Value="1.13657023776269738846e-01"/>
                <Y Unit="in" Value="2.10778191616462606817e+01"/>
                <Z Unit="in" Value="4.04163897920995651702e+01"/>
            </Origin>
        </TransformationMatrix>
    </Base>
</Spring>
\end{lstlisting}

Tag-urile Action si Base de la elementul Joint și Connector arată corpurile de miscare legate.

Toate elementele au o legătura strânsa iar buna funcționalitate a aplicației tine de structura fișierului.

\subsection{Turtle graphics}
\subsubsection{Desenarea unui cilindru}
\begin{lstlisting}
void JointPainterPathCreator::drawRevolutePath(
    Turtle & turtle, 
    double length
    ) const
    
    const double radius = 5;

    turtle.forward((length / 2) - radius);

    turtle.save();
    
    //drawing first circle
    turtle.rotate(M_PI / 2);
    const int nrOfSegments = 32;
    const double angle = 2 * M_PI / nrOfSegments;    
    const double segmentLength = radius*sqrt(2.0 - 2.0*cos(angle));

    for (int i = 0; i < nrOfSegments; i++) {
        turtle.forward(segmentLength);
        turtle.rotate(-angle);
    }

    turtle.load();
    turtle.forward(2 * radius, false);

    //drawing cilinder upper line
    const double cilinderLength=10;

    turtle.forward(cilinderLength, false);

    turtle.rotate(-M_PI / 2);
    turtle.forward(radius, false);

    turtle.rotate(-M_PI / 2);
    turtle.forward(cilinderLength + 3);
    turtle.rotate(M_PI);
    turtle.forward(cilinderLength + 3, false);

    turtle.save();
    for (int i = 0; i < nrOfSegments / 2; i++) {
        turtle.forward(segmentLength);
        turtle.rotate(angle);
    }

    turtle.load();
    turtle.rotate(M_PI / 2);
    turtle.forward(radius*2.0,false);
    turtle.rotate(M_PI / 2);

    //drawing cilinder lower line
    turtle.forward(cilinderLength + 3);
    turtle.rotate(M_PI);
    turtle.forward(cilinderLength + 3, false);
    turtle.rotate(-M_PI);

    turtle.rotate(M_PI / 2);
    turtle.forward(radius, false);

    turtle.rotate(M_PI / 2);
    turtle.forward(radius, false);
}
}
\end{lstlisting}
