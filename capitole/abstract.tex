\newpage
\section{Introducere}


În cartografie sau geologie, o hartă topologică este un tip de diagramă care conține doar informații esențiale. 
Numele este derivat din topologie, o ramură a matematicii care studiază proprietățile unui obiect care nu se schimba dacă 
obiectul în sine este deformat, prin întindere, îndoire, dar nu și rupere sau lipire. Aceste hărți nu au scară, iar distanța 
și direcția sunt supuse schimbării și variației, relația dintre puncte esentiale este menținută fiind caracteristica de bază. 
Un exemplu ar fi harta unui metrou care conține doar date importante precum numele stațiilor și ordinea lor, însa distanța 
dintre ele nu este întocmai cea din realitate. \newline

Acest concept este aplicat pentru crearea unei diagrame 2D având la baza un model 3D format din mai multe elemente.
Modelul 3D este generat de aplicație de simulare Simcenter sub forma unui fisier .mdef care contine doar informații esențiale.
Aplicația prezentată în această lucrare are scopul de a vizualiza doar noțiuni semnificative precum relațiile de legatură dintre elemente,
și tipul lor. Această aplicație este necesară în cazurile când nu se dorește deschiderea întregului model în aplicație 
(lucru ce se dovedește ineficient în unele cazuri din punct de vedere al timpului de încărcare), ci doar crearea unei diagrame simple 
conținând caracteristicile prezentate anterior.\newline
