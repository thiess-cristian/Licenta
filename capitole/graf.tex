\newpage
\section{Graf}
Un graf este o structura formata din obiecte în care sunt puse în evidenta legăturile dintre acele obiecte. 
Obiectele corespund unor abstracții matematice numite într-un graf noduri/vârfuri (numite și puncte) și fiecare legătură 
dintre perechile de obiecte asociate se numește muchie (numită și arc sau linie, prin care este și reprezentată). 
De obicei, un graf este reprezentat în formă schematică ca un set/grup de puncte pentru noduri, iar aceste sunt unite două 
câte două de linii sau curbe pentru muchii. Grafurile reprezintă unul dintre obiectele de studiu în matematica discretă. 
Muchiile pot fi orientate/directe sau neorientate/nedirecte.\newline

Din punct de vedere matematic un graf este o pereche \(G=(V,E)\), unde \(V\) este un set de elemente numite noduri, iar \(E\) reprezinta un set de perechi de noduri, numite noduri.
Un graf vid este un graf în care mulțimea nodurilor \(V\) este vida (implicit și mulțimea muchiilor \(E\) este vida).
Ordinul unui graf reprezinta numărul de noduri, notat cu \(|V|\). Mărimea unui graf este numărul de muchii \(|E|\).
Gradul unui nod este numărul de muchii care sunt incidente cu el.
Într-un graf de ordin \(n\), gradul maxim al oricărui nod este de \(n-1\), iar numărul maxim de muchii este \(n(n-1)/2\).

\subsection{Multigraf}

Un multigraf este o generalizare în care oricare doua noduri pot avea mai multe muchii intre ele. Acele muchii sunt numite și muchii paralele sau muchii multiple.
Exista doua noțiuni distincte despre muchii paralele:\newline

\begin{itemize}
\item Muchii fără identitate: identitatea unei muchii tine de cele doua noduri de care aparține. 
In unele cazuri nevoia de a distinge muchii multiple dintre doua noduri poate lipsi, iar acele muchii sunt considerate ca o 
singura entitate.\newline
\item Muchii cu identitate: în acest caz fiecare muchie este considerata ca fiind o primitiva, la fel ca nodurile, 
iar în cazul în care intre doua noduri exista mai multe muchii, fiecare dintre ele este considerata fiind o entitate distincta.\newline
\end{itemize}

Definitie matematica:\newline

\begin{itemize}
\item Multigraf neorientat, cu muchii fara identitate: 
\(G=(V,E)\) unde:\newline 
\(V\) este un set de noduri \newline 
\(E\) este un multiset de perechi de noduri, numite muchii.
    
\item Multigraf neorientat, cu muchii cu identitate:
\(G=(V,E,r)\), unde:\newline
\(V\) este un set de noduri\newline
\(E\) este un set de muchii\newline
\(r : E → \{\{x,y\} : x, y \in V\}\)

\end{itemize}

\subsection{Graf planar}

Un graf planar este un graf care poate fi incorporat într-un plan, adică un graf care poate fi desenat astfel încât 
marginile sale sa se intersecteze doar în noduri. Cu alte cuvinte, muchiile grafului sa nu se suprapună.\newline

Matematicianul polonez Kazimierz Kuratowski a caractetizat idea de graf planar prin următoarea teoremă:
Un graf finit este planar dacă și numai dacă nu conține un subgraf care este o subdiviziune a grafului complet \(K_5\) sau a grafului complet bipartit \(K_{3,3}\).
O subdiviziune a unui graf rezulta din inserarea a oricâte noduri într-o muchie.\newline

O alta modalitate de a descrie un graf planar este prin teorema lui Wagner, exprimata în legătura cu noțiunea de graf minor:
Un graf finit este planar dacă și numai dacă și numai dacă nu conține grafurile \(K_5\) sau \(K_{3,3}\) ca graf minor.
Un minor al unui graf rezulta prin contractarea unei muchii într-un nod, fiecare vecin al nodului original devenind vecin cu nodul nou.\newline

Alte criterii de planaritate:\newline
In practica este dificil sa folosim teorema lui Kuratowski pentru a decide într-un mod eficient dacă un graf este planar sau nu. 
Exista și alți algoritmi pentru rezolvarea acestei probleme, pentru un graf cu n noduri cu o complexitate de \(O(n)\).

Un graf simplu  cu v>=3 noduri, e muchii și f fete, trebuie sa îndeplinească următoarele condiții pentru a fi planar:\newline
Teorema 1: e<=3v-6 \newline
Teorema 2: dacă nu sunt cicluri de lungime 3, atunci e<=2v-4 \newline
Teorema 3: f<=2v-4 \newline

Formula lui Euler: \newline
Formula lui Euler afirmă că dacă un graf planar, finit, conectat, este desenat într-un plan și v este numărul de vârfuri, 
e este numărul de muchii și f este numărul de fețe (regiuni marcate de muchii , inclusiv regiunea exterioară, infinit de mare)
\[v-e+f=2\]

Ca o ilustrare, în graful fluture dat mai sus, v = 5, e = 6 și f = 3. În general, dacă proprietatea este valida 
pentru toate grafurile planare cu f fete, orice modificare a grafului care ar crea o față suplimentară, graful 
fiind în continuare planar ar fi și v - e + f invariant. Din moment ce proprietatea este valid pentru toate grafurile 
cu f = 2, prin inducție matematică este valabilă pentru toate cazurile. Formula lui Euler poate fi demonstrata și în 
modul următor: dacă graful nu este un arbore, șterge o muchie care completează un ciclu. Astfel scade atât e, cât și 
f cu unul, lăsând v - e + f constantă. Repetați până când graficul rămas este un arbore; copacii au v = e + 1 și f = 1, 
producând v - e + f = 2.\newline

Într-un graf planar, simplu, conectat, simplu, orice față (cu excepția celei exteriorare) este marginita de cel puțin trei 
muchii și fiecare muchie se afla intre cel mult 2 fete; folosind formula lui Euler, se poate arăta apoi că aceste grafuri  
sunt sparse în sensul că dacă v ≥ 3: 
\[e<=3v-6\]

\subsection{Aplicare}

Grafurile pot fi folosite pentru modelarea relațiilor și proceselor într-un sistem fizic, biologic, 
social sau informatic. Multe probleme practice pot fi reprezentate prin grafuri. In domeniul informaticii 
grafurile pot fi folosite pentru a reprezenta rețele de comunicare, organizarea datelor, dispozitive computerizate, 
fluxul de calcul, etc. De exemplu și un site web poate fi reprezentat printr-un graf, paginile web fiind nodurile 
iar legăturile dintre ele, link-urile, fiind muchiile. O abordare similara poate fi proiectate pentru multe alte domenii, 
precum ingineria mecanica unde putem exprima relațiile dintre mai multe elemente mecanice printr-un graf. Dezvoltarea 
algoritmilor pentru a gestiona grafuri este, prin urmare, de interes major în domeniul informaticii.










