\newpage
\section{Istorie}
Topologia, ca și o ramura  bine definita a matematicii, își are originile de la începutul secolului XX, 
însa au fost descoperite și cazuri izolate în aplicarea acestui concept chiar și cu câteva secole în urma. 
Leonhard Euler este considerat ca fiind primul care abordează acest domeniu prin lucrarea sa scrisa în 1736, 
Cele șapte poduri din Konigsberg, care se refera la crearea unui traseu prin oraș trecând peste fiecare pod o 
singura data., lucru ce sa dovedit imposibil.\newline

La 14 noiembrie 1750, Euler a scris unui prieten că a realizat importanța marginilor unui poliedru. 
Aceasta a condus la formula lui poliedrala: \[V-E+F=2\] unde V, E și F indică numărul de vârfuri, 
muchii și fețe ale poliedrului. Această analiză este considerata drept prima teoremă, semnalizând nașterea 
topologiei.\newline

Alte contribuții au fost făcute de Augustin-Louis Cauchy, Ludwig Schläfli, Johann Benedict Listing, 
Bernhard Riemann și Enrico Betti. Listing a introdus termenul "Topologie" în "Vorstudien zur Topologie", 
scris în limba germană, în 1847.\newline 
