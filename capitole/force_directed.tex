\newpage
\section{Algortimi force-directed de desenare a grafurilor}

Este o clasa de algoritmi  de desenare a grafurilor într-un mod plăcut din punct de vedere estetic. 
Prin acest algoritm se positioneaza nodurile grafului într-un spațiu bidimensional sau tridimensional astfel încât 
lungimea muchiilor este mai mult sau mai puțin egala iar în structura grafului exista un număr minim de suprapuneri intre muchii, 
iar în cazuri favorabile nici o suprapunere. Desenarea grafurilor poate fi o problema grea de abordat dar prin algoritmii 
de desenare force-directed, se exclud noțiuni ce țin de planaritatea grafurilor, întreaga operație de desenare fiind mai 
degrabă o simulare fizica.\newline

Prin asignarea anumitor forte setului de  noduri și setului de muchii  din graf acestea fie se apropie fie se depărtează după 
fiecare iterație a algoritmului, scopul algoritmului fiind de a minimiza deplasarea nodurilor sau a muchiilor pana când se 
ajunge la o stare de echilibru. De obicei forte elastice bazate pe legea lui Hooke sunt atribuite nodurilor unei muchii 
pentru ca acestea sa se atragă, în schimb ce simultan forte de repulsie, similare cu cele a particulelor cu sarcina electrica 
bazate pe legea lui Coulomb, sunt folosite pentru distanțarea nodurilor. In stările de echilibru al acestui sistem de forte, 
muchiile au de obicei lungimea egala (datorita forței elastice de atracție), iar nodurile care nu sunt conectare printr-o 
muchie se resping intre ele (datorita forței de repulsie). Fortele de atracție și repulsie pot fi definite și prin funcții 
care nu bazate pe legile fizice ale elasticității sau sarcinii electrice, de exemplu unele sisteme pot folosi funcții de 
atracție logaritmice în loc de funcții liniare.\newline

O forță similara cu cea a gravitații poate fi folosita pentru a atrage nodurile către un punct fix în planul de desenare, 
astfel pentru un graf cu mai multe componente conexe acestea vor sta relativ în aceeași zona și nu se vor mai depărta 
constant una de cealaltă datorita forței continue de repulsie și a lipsei de atracție intre ele. Forte similare cu cele 
magnetice pot fi aplicate într-un sistem, atât pe noduri cat și pe muchii în sine pentru a evita suprapunerea muchiilor 
în întreaga simulare.\newline

\subsection{Aplicare}

Odată ce forțele au fost atribuite nodurilor sau muchiilor dintr-un graf întregul comportament al elementelor 
este simulat ca într-un sistem fizic unde nodurile se deplasează la fiecare iterație. Acest proces este repetat 
iterativ pana când se ajunge la o stare de echilibru mecanic, adică pozițiile relative ale elementelor nu se mai 
schimba de la o iterație la alta. Structura finala după întregul proces este folosita pentru a reprezenta graful ce 
trebuie desenat.\newline

Este posibila și adăugarea unui mecanism de calculare a stării de echilibru pe lângă simularea fizica, acestea fiind 
metode de optimizare globala precum algoritmi genetici.\newline

\subsection{Avantaje}
\begin{itemize}
\item Rezultate calitative

Pentru grafuri de mărimi medii (50-500) de noduri de multe ori rezultatul aplicării unui algoritm force directed este 
unul bun și inteligibil din perspectiva omului. Rezultatul fiind catalogat după următoarele criterii: lungimea uniforma 
a muchiilor, distribuirea uniforma a nodurilor, simetrie. Simetria este un criteriu greu de îndeplinit și depinde de 
structura interna a grafului.
\item Flexibilitate

Algoritmul ales poate fi adaptat pentru a îndeplinii criterii adiționale de desenare. 
Exemple de extensibilitate are fi desenarea de grafuri orientate, grafuri 3D, grafuri împărțite în clustere.
\item Intuitivitate

Având la baza analogii din fizica, precum elasticitatea, comportamentul algoritmului este ușor de prezis și înțeles.
\item Simplitate

De obicei astfel de algoritmi sunt ușor de implementat, sau au fost deja implementati in anumite biblioteci.
\item Interactivitate

Prin desenarea unor stagii intermediare a grafului pe care este aplicat algoritmul utilizatorul poate urmării evoluția 
procesului, observând cum se dezvolta dintr-o structura încurcata cu multe suprapuneri de muchii, într-un graf echilibrat 
și ușor de urmărit.
\item Fundatie teoretica 

Inca din anii 1960 metode de desenare a grafurilor au fost cercetate, o prima implementare fiind făcuta 
prin algoritmul lui Tutte în anul 1963 folosind o reprezentare baricentrica (prin care se calcula un nod 
de baza care reprezenta centrul de greutate a întregului graf), folosind doar forte de atracție. 
In următorii ani alte metode de desenare au fost create de către Eades(1984), Fruchterman \& Reingold (1991), Kamada \& Kawai (1989).

\end{itemize}

\subsection{Dezavantaje}

\begin{itemize}
\item Timp de rulare mare

De obicei complexitatea unui algoritm de desenare force-directed este \(O(n^3)\), unde n este numărul de noduri din graf. 
Acest lucru rezulta deoarece numărul iterațiilor este estimat ca fiind \(O(n)\) iar pentru fiecare iterație fiecare 
pereche de noduri trebuie vizitata pentru calcularea forței de repulsie mutuale.

\item Minim local

Prin algoritmii force-directed se ajunge la o structura în care nodurile se afla într-o stare de echilibru, întreg procesul se 
poate considera ca fiind rezolvarea unei probleme de optimizare.
De cele mai multe ori starea rezultata se poate afla într-un minim local al funcție ce trebuie optimizate, dar care nu 
coincide neapărat cu un minim global. Intreg procesul depinde de starea inițială a nodurilor care este aleasa aleatoriu. 
Aceasta problema creste odată cu numărul de noduri din graf. O modalitate de rezolvare ar fi combinarea a mai multor 
algoritmi force-directed pentru obținerea unui rezultat mai bun. De exemplu folosirea algoritmului Kamada–Kawai pentru 
crearea unei așezări în plan inițiale, iar apoi îmbunătățirea structurii cu ajutorul algoritmului Fruchterman–Reingold.

\end{itemize}

\subsection{Metode de desenare}
\subsubsection{Metoda baricentrica}
Metoda baricentrica de desenare a lui Tutte din 1963 este considerata prima versiune a unui algoritm force-directed 
pentru obținerea unei structuri cu muchii drepte, fără suprapuneri pentru un graf planar 3-conex. In comparație cu alte 
metode de desenare aceasta varianta garantează ca fetele ale grafului planar sunt convexe. Idea în spatele algoritmului 
consta în faptul ca dacă o fata a grafului planar este fixata în plan, atunci pozițiile potrivite a celorlalte noduri sunt 
găsite rezolvant un set de ecuații liniare, unde fiecare poziția unui nod este reprezentata ca o combinație convexa a 
pozițiilor nodurilor vecine lui. In acest model forța unei muchii (u,v) este direct proporțională cu distanta dintre 
nodul u și v in plan. Astfel forța la un nod v este descrisa prin F(v)=sum(pu-pv), oricare (u,v) din E, unde pu și pv 
sunt pozițiile nodurilor. Deoarece aceasta funcție are un minim cu toate nodurile așezate în aceeași locație, setul 
de noduri trebuie împărțit în noduri fixe și noduri libere, care pot fi deplasate.

TODO:alg

\subsubsection{Algoritmul Fruchterman–Reingold}

Algoritmul Fruchterman and Reingold introduce noțiunea de temperatura. Aceasta variabila are o valoare setata 
la început iar ea scade pana la fiecare iterație pana la 0. Temperatura controlează deplasarea nodurilor iar cu 
cat ne apropiem de un minim, cu atât temperatura v-a fi mai mica pentru nu a schimba cu mult rezultatul.


