\newpage
\section{Algortimi force-directed de desenare a grafurilor}

Este o categorie de algoritmi de desenare a grafurilor într-un mod plăcut din punct de vedere estetic. 
Printr-un asftle de algoritm se positioneaza nodurile grafului într-un spațiu bidimensional sau tridimensional astfel încât 
lungimea muchiilor este mai mult sau mai puțin egală iar în structura grafului există un număr minim de suprapuneri intre muchii, 
iar în cazuri favorabile nici o suprapunere. Desenarea grafurilor poate fi o problema grea de abordat dar prin algoritmii 
de desenare force-directed, se exclud noțiuni ce țin de planaritatea grafurilor, întreaga operație de desenare fiind mai 
degrabă o simulare cu aspecte ce se regăsesc în fizică.\newline

Prin asignarea anumitor foțe setului de  noduri și setului de muchii din graf acestea fie se apropie fie se depărtează după 
fiecare iterație a algoritmului, scopul algoritmului fiind de a minimiza deplasarea nodurilor sau a muchiilor pana când se 
ajunge la o stare de echilibru. De obicei forțe elastice bazate pe legea lui Hooke sunt atribuite nodurilor unei muchii 
pentru ca acestea sa se atragă, în schimb ce simultan forte de repulsie, similare cu cele a particulelor cu sarcina electrică 
bazate pe legea lui Coulomb, sunt folosite pentru a crea o repulsie între nodurile grafului. In stările de echilibru al acestui sistem de forte, 
muchiile nodurilor care au un parinte comun au de obicei lungimea aproape egala (datorita forței elastice de atracție), 
iar nodurile care nu sunt conectare printr-o muchie se resping intre ele (datorita forței de repulsie). Fortele de atracție și repulsie pot fi definite și prin funcții 
care nu sunt bazate pe legile fizice ale elasticității sau sarcinii electrice, de exemplu unele sisteme pot folosi funcții de 
atracție logaritmice în loc de funcții liniare.\newline

O forță similară cu cea a gravitații poate fi folosită pentru a atrage nodurile către un punct fix în planul de desenare, 
astfel pentru un graf cu mai multe componente conexe acestea vor sta relativ în aceeași zonă și nu se vor mai depărta 
constant una de cealaltă datorita forței continue de repulsie și a lipsei de atracție intre ele. Forțe similare cu cele 
magnetice pot fi aplicate într-un sistem, atât pe noduri cât și pe muchii în sine pentru a evita suprapunerea muchiilor 
în întreaga simulare.\newline

\subsection{Aplicare}

Odată ce forțele au fost atribuite nodurilor sau muchiilor dintr-un graf întregul comportament al elementelor 
este simulat ca într-un sistem fizic unde nodurile se deplasează la fiecare iterație. Acest proces este repetat 
iterativ pana când se ajunge la o stare de echilibru mecanic, adică pozițiile relative ale elementelor nu se mai 
schimba de la o iterație la alta. Structura finala după întregul proces este folosită pentru a reprezenta graful ce 
trebuie desenat.\newline

Este posibilă și adăugarea unui mecanism de calculare a stării de echilibru pe lângă simularea fizică, acestea fiind 
metode de optimizare globala precum algoritmi genetici.\newline

\subsection{Avantaje}
\begin{itemize}
\item Rezultate calitative

Pentru grafuri de mărimi medii (50-500) de noduri de multe ori rezultatul aplicării unui algoritm force-directed este 
unul bun și inteligibil din perspectiva omului. Rezultatul fiind catalogat după următoarele criterii: 
\begin{itemize}
    \item lungimea uniforma a muchiilor
    \item distribuirea uniforma a nodurilor
    \item numar de supirapuneri minim
    \item simetrie
\end{itemize}

Simetria este un criteriu greu de îndeplinit și depinde de structura interna a grafului.
\item Flexibilitate

Algoritmul ales poate fi adaptat pentru a îndeplinii criterii adiționale de desenare. 
Exemple de extensibilitate are fi desenarea de: 
\begin{itemize}
    \item grafuri orientate
    \item grafuri 3D
    \item grafuri împărțite în clustere
    \item multigrafuri
    \item grafuri ponderate
\end{itemize}
\item Intuitivitate

Având la baza analogii din fizica, precum elasticitatea, comportamentul algoritmului este ușor de prezis și înțeles.
\item Simplitate

De obicei astfel de algoritmi sunt ușor de implementat, sau au fost deja implementati in anumite biblioteci.
\item Interactivitate

Prin desenarea unor stagii intermediare a grafului pe care este aplicat algoritmul, utilizatorul poate urmării evoluția 
procesului, observând cum se dezvolta dintr-o structura încurcata cu multe suprapuneri de muchii, într-un graf echilibrat 
și ușor de urmărit.
\item Fundatie teoretica 

Inca din anii 1960 metode de desenare ale grafurilor au fost cercetate, o prima implementare fiind făcuta 
prin algoritmul lui Tutte în anul 1963 folosind o reprezentare baricentrică (prin care se se seteaza un grup de noduri 
de baza în jurul cărora celelalte noduri vor fi plasate), folosind doar forte de atracție. 
In următorii ani alte metode de desenare au fost create de către Eades(1984), Fruchterman și Reingold (1991), Kamada și Kawai (1989).

\end{itemize}

\subsection{Dezavantaje}

\begin{itemize}
\item Timp de rulare mare

De obicei complexitatea unui algoritm de desenare force-directed este \(O(n^3)\), unde \(n\) este numărul de noduri din graf. 
Acest lucru rezulta deoarece numărul iterațiilor este estimat ca fiind \(O(n)\) iar pentru fiecare iterație fiecare 
pereche de noduri trebuie vizitata pentru calcularea forței de repulsie mutuale.

\item Minim local

Prin algoritmii force-directed se ajunge la o structura în care nodurile se afla într-o stare de echilibru, întreg procesul poate 
fi consideratca ca fiind rezolvarea unei probleme de optimizare.
De cele mai multe ori starea rezultata se poate afla într-un minim local al funcție ce trebuie optimizate, dar care nu 
coincide neapărat cu un minim global. Întreg procesul depinde de starea inițială a nodurilor care este aleasa aleatoriu. 
Aceasta problema creste odată cu numărul de noduri din graf. O modalitate de rezolvare ar fi combinarea a mai multor 
algoritmi force-directed pentru obținerea unui rezultat mai bun. De exemplu folosirea algoritmului Kamada–Kawai pentru 
crearea unei așezări în plan inițiale, iar apoi îmbunătățirea structurii cu ajutorul algoritmului Fruchterman–Reingold.

\end{itemize}

\subsection{Metode de desenare}
\subsubsection{Metoda baricentrică}
Metoda baricentrică de desenare a lui Tutte din 1963 este considerată prima versiune a unui algoritm force-directed 
pentru obținerea unei structuri cu muchii drepte, fără suprapuneri pentru un graf planar 3-conex. In comparație cu alte 
metode de desenare această varianta garantează ca fețele grafului planar sunt convexe. Idea în spatele algoritmului 
consta în faptul ca dacă o față a grafului planar este fixată în plan, atunci pozițiile potrivite a celorlalte noduri sunt 
găsite rezolvănd un set de ecuații liniare, unde fiecare poziția a unui nod este reprezentata ca o combinație convexa a 
pozițiilor nodurilor vecine lui. In acest model forța unei muchii \((u,v)\) este direct proporțională cu distanta dintre 
nodul \(u\) și \(v\) in plan. Astfel forța la un nod \(v\) este descrisa prin 
\[F(v)=\sum_{(u,v) \in E} p_u-p_v\] 
unde \(p_u\) și \(p_v\) sunt pozițiile nodurilor. Deoarece aceasta funcție are un minim cu toate nodurile așezate în aceeași locație, setul 
de noduri trebuie împărțit în noduri fixe și noduri libere, care pot fi deplasate.

\begin{algorithm}[H]
    \caption{Metoda baricentrică de desenare}
    Input: \(G=(V,E)\);\newline
    partiționarea lui \(V\) intr-un set \(V_0\) cu cel puțin 3 noduri fixe și un set \(V_1\) cu noduri libere, \(V=V_0 \cup V_1\);\newline
    un poligon convex \(P\) cu \(|V_0|\) vârfuri.\newline

    Output: o poziție \(p_v\) pentru fiecare nod din \(V\) astfel încat nodurile fixe formeaza un poligon.
        
    \begin{algorithmic}[1]
        \State Plaseaza fiecare nod \(u \in V_0\) la un varf al lui \(P\), și fiecare nod liber în origine
        \Repeat
        \ForEach{ \(v \in V_1 \) }
            \State \(x_v=\frac{1}{deg(v)} \sum_{(u,v) \in E} x_u\)
            \State \(y_v=\frac{1}{deg(v)} \sum_{(u,v) \in E} y_u\)
        \EndFor
        \Until{ \(x_v\) și \(y_v\) converg pentru toate nodurile libere \(v\)}
    \end{algorithmic}
\end{algorithm}

\subsubsection{Algoritmul Fruchterman–Reingold}

Algoritmul Fruchterman-Reingold de desenare a grafurilor introduce noțiunea de temperatură. Această variabilă are o valoare setată 
la început iar ea scade la fiecare iterație pana la \(0\). Temperatura controlează deplasarea nodurilor iar cu 
cât ne apropiem de un minim, cu atât temperatura v-a fi mai mică pentru a nu schimba cu mult rezultatul.

\begin{algorithm}[H]
    \caption{Fruchterman si Reingold}
    \(arie=W*L\); \(W\) si \(L\) sunt lațimea si lungimea suprafeței de desenare\newline
    \(G=(V,E)\); nodurile au poziții inițiale aleatorii\newline
    \(k=\sqrt{arie/|V|}\) \newline
    \(f_a(x)=x^2/k\)\newline
    \(f_r(x)=k^2/x\)\newline 
    \begin{algorithmic}[1]
        \For{\(i=1\) to \(iteratii\)}
            \State calcularea forțelor de repulsie
            \ForEach{\(v \in V\)}
                \State fiecare nod are doi vectori \(pos\) și \(disp\)
                \State \(v.disp=0\)
                \ForEach{\(u \in V\)}
                    \If{ \(u \neq v\)}
                        \State \(\delta\) este vectorul diferență dintre pozițiile celor doua noduri
                        \State \(\delta=v.pos-u.pos\)
                        \State \(v.disp=v.disp+(\delta/|\delta|)*f_r(|\delta|)\)
                    \EndIf
                \EndFor
            \EndFor

            \State calcularea forțelor de atracție
            \ForEach{\(e \in E\)}
                \State fiecare muchie este o perche de noduri \(u\) și \(v\)
                \State \(\delta=e.v.pos-e.u.pos\)
                \State \(e.v.disp=e.v.disp-(\delta/|\delta|)*f_a(|\delta|)\)
                \State \(e.u.disp=e.u.disp+(\delta/|\delta|)*f_a(|\delta|)\)
            \EndFor

            \State deplasarea maxima se limitează în funcție de temperatura \(t\)
            \ForEach{\(v \in V\)}
                \State \(v.pos=v.pos+(v.disp/|v.disp|)*min(v.disp,t)\)
                \State \(v.pos.x=min(W/2,max(-W/2,v.pos.x))\)
                \State \(v.pos.y=min(L/2,max(-L/2,v.pos.y))\)
            \EndFor

            \State reducerea temperaturii 
            \State \(t=cool(t)\)
        \EndFor
    \end{algorithmic}
\end{algorithm}

  

