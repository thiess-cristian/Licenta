\newpage
\section{Tehnologii folosite}
\subsection{C++}

C++ este un limbaj de programare creat de Bjarne Stroustrup ca o extensie a limbajului C, inițial fiind numit “C cu clase”. Limbajul s-a extins semnificativ, C++ modern având caracteristici orientate pe obiect, de programare generica și funcțională precum și accesul la manipularea memoriei. Este aproape întotdeauna implementat ca limbaj compilat, iar mulți furnizori oferă compilatoare C ++, inclusiv LME, LLVM, Microsoft, Intel și IBM, astfel încât limbajul sa poată fi compilat pe mai multe platforme. 

A fost conceput cu o înclinare către programarea sistemelor și programarea embeded, software limitat de resurse și sisteme mari, punând în prim plan performanta, eficienta și flexibilitatea ca și caracteristici esențiale ale limbajului. C++ s-a dovedit fiind util și în alte contexte precum aplicații desktop, servere, centrali telefonice.

C ++ este standardizat de către Organizația Internațională pentru Standardizare (ISO), cu cea mai recentă versiune standard publicată de ISO în decembrie 2017 ca ISO / IEC 14882: 2017 (informal cunoscut sub numele de C ++ 17). Înainte de standardizarea inițială din 1998, C ++ a fost dezvoltat de omul de știință danez Bjarne Stroustrup la Bell Labs din 1979, ca extensie a limbajului C; el dorea un limbaj eficient și flexibil similar C, care să ofere și caracteristici de nivel înalt pentru organizarea programelor.
