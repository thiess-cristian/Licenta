\newpage
\section{Proiectie ortografica}

Proiecția ortografica (sau proiecția ortogonala) este o modalitate de a reprezenta un model tridimensional  
într-un spațiu bidimensional. Este o forma de proiecție paralela, în care toate liniile de proiecție sunt ortogonale 
cu planul de proiecție.\newline

Termenul ortografic este uneori rezervat în mod specific pentru reprezentări ale obiectelor în care axele principale 
sau planurile obiectului sunt de asemenea paralele cu planul de proiecție, dar acestea sunt mai bine cunoscute 
ca proiecții multiview. Mai mult, atunci când planurile sau axele principale ale unui obiect într-o proiecție 
ortografică nu sunt paralele cu planul de proiecție, dar sunt înclinate mai degrabă pentru a descoperi mai 
multe laturi ale obiectului, proiecția se numește o proiecție axonometrică. Subtipurile de proiecție multiview 
includ planuri, elevații și secțiuni. Subtipurile de proiecții axonometrice includ proiecții izometrice, dimetrice și trimetrice. \newline

O proiecte ortografica simpla, pe planul \(z=0\) poate fi definita de urmatoarea matrice:

\[
P=
  \begin{bmatrix}
    1 & 0 & 0 \\
    0 & 1 & 0 \\
    0 & 0 & 0 
  \end{bmatrix}
\]

Pentru fiecare punct \(v=(v_x,v_y,v_z)\), punctul transformat ar fi:

\[
P_v=
  \begin{bmatrix}
    1 & 0 & 0 \\
    0 & 1 & 0 \\
    0 & 0 & 0 
  \end{bmatrix}
  \begin{bmatrix}
    v_x  \\
    v_y  \\
    v_z  
  \end{bmatrix}
  =
  \begin{bmatrix}
    v_x  \\
    v_y  \\
    0  
  \end{bmatrix}
\]


