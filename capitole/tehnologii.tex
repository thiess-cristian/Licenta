\newpage
\section{Tehnologii folosite}
\subsection{C++}

C++ este un limbaj de programare creat de Bjarne Stroustrup ca o extensie a limbajului C, inițial fiind numit "C cu clase". 
Limbajul s-a extins semnificativ, C++ modern având caracteristici orientate pe obiect, de programare generica și funcțională 
precum și accesul la manipularea memoriei. Este de obicei implementat ca limbaj compilat, iar mulți furnizori oferă 
compilatoare C ++, inclusiv LME, LLVM, Microsoft, Intel și IBM, astfel încât limbajul sa poată fi compilat pe mai multe platforme.\newline 

A fost conceput cu o înclinare către programarea sistemelor și programarea embeded, pentru software limitat de resurse și sisteme mari, 
punând în prim plan performanța, eficiența și flexibilitatea ca și caracteristici esențiale ale limbajului. C++ s-a dovedit 
fiind util și în alte contexte precum aplicații desktop, servere, centrale telefonice.\newline

C++ este standardizat de către Organizația Internațională pentru Standardizare (ISO), cu cea mai recentă versiune standard 
publicată de ISO în decembrie 2017 ca ISO / IEC 14882: 2017 (informal cunoscut sub numele de C++17). Înainte de standardizarea 
inițială din 1998, C++ a fost dezvoltat de Bjarne Stroustrup la Bell Labs din 1979, ca extensie a 
limbajului C; el dorea un limbaj eficient și flexibil similar C, care să ofere și caracteristici de nivel înalt pentru organizarea 
programelor.\newline

\subsection{Qt}

Qt este un toolkit open-source și gratis folosit la crearea de interfețe grafice pentru aplicații cross-platform care 
pot fi rulate pe diverse platforme hardware si software precum Linux, Windows, macOS, Adroid sau sisteme embeded, 
cu puține sau fară schimbari in codul de baza dar păstrând funcționalitățile native. Qt este momentan dezvoltat 
de The Qt Company sub Qt Project prin guvernare open-source, în care sunt implicate mai multe organizații și dezvoltatori 
individuali cu scopul de aduce noi funcționalități librăriei Qt.\newline

Majoritatea programelor GUI dezvoltate cu Qt au o interfață nativa similara cu cea a sistemului pe care este rulat. 
Aplicații precum unelte ce pot fi rulate din linia de comandă sau console pentru servere, fără interfață grafica pot fi 
dezvoltate în Qt. Un exemplu fiind framework-ul web Cutelyst.\newline

Qt suportă diverse compilere precum GCC C++ și MSVC de la Visual Studio. Qt conține și Qt Quick, care include propriul 
limbaj de scriptare numit QML, care a facilitat dezvoltarea rapida a aplicațiilor mobile păstrând opțiunea de a scrie 
logică în C++ pentru cele mai bune rezultate în legătura cu performanta.\newline

Alte funcționalități includ acces la baze de date prin SQL, cititor de XML, cititor de JSON, managementul thread-urilor.\newline

Qt este construit pe aceste concepte:
\begin{itemize}
    \item Abstractizarea interfeței grafice
    
    La început Qt folosea propriul engine de desenare pentru a emula aspectul diferitelor platforme de pe care rula. 
    Astfel doar câteva clase depindeau de platforma făcând portarea mult mai ușoară, însa emularea nu era mereu perfecta. 
    Versiunile mai noi de Qt folosesc API-uri native de pe platforma de rulare, pentru a rezolva problemele din trecut.

    \item Signals si slots
    
    Este un concept introdus în Qt pentru comunicarea între obiecte, având la baza conceptul șablonului de proiectare observer 
    evitând cod boilerplate. Obiectele ce țin GUI pot transmite semnale ce conțin informații relative la un eveniment care pot 
    fi captate de alte obiecte care conțin socluri.

    \item Compilator de metaobiecte

    Numit și moc, este o unealta care rulează pe codul unei aplicații Qt. Interpretează anumite macro-uri scrise în codul 
    sursa C++ și le folosește pentru a genera cod C++ cu ajutorul unor informații legate de clasele din aplicație. 
    Acest concept este folosit pentru a adăuga funcționalitati care nu sunt prezente nativ în limbajul C++: semnale și socluri, 
    introspecție, apelări asincrone de funcții.

    \item Legături intre limbaje

    Qt poate fi folosit și de alte limbaje precum Python, Javascript, C\# sau Rust.
\end{itemize}

\subsubsection{Graphics view framework}

Graphics View pune la disoziție un mediu în care se pot controla și se poate interacționa cu un număr mare de elemente grafice 2D. 
Framework-ul include și o arhitectura de propagarea a evenimentelor care facilitează interacțiunea cu elementele din scena. 
Elementele suporta evenimente legate de taste, mișcarea mouse-ul, evenimente de dublu click, etc. \newline

Graphics View folosește un arbore BSP (Binary Space Partitioning) care oferă o accesare rapida la oricare element, astfel se poate 
vizualiza și interacționa cu scene cu milioane de elemente grafice în timp real.\newline

Graphics View furnizează o abordare bazate pe elemente a programării model-view. Mai multe view-uri pot observa o scena, 
iar scena conține elementele grafice compuse dintr-o varietate de forme geometrice.\newline

\begin{itemize}
    \item Scena
    
    Clasa QGraphicsScene oferă următoarele funcționalități:
    \begin{itemize}
        \item O interfață pentru controlarea unui număr mare de elemente grafice.
        \item Propagarea evenimentelor din exterior către elemente.
        \item Setarea stării unui elemente, ex: selectie, focus.
    \end{itemize}

    Scena reprezinta un container pentru obiecte de tip \verb|QGraphicsScene|. Elementele sunt adăugate prin funcția 
    \verb|QGraphicsScene::addItem()|, iar apoi pot fi accesate folosind una din multele funcții de returnare.Functia 
    \verb|QGraphicsScene::items()| și supraincarcariile ei, returnează toate elementele grafice care se intersectează sau 
    care sunt conținute de un punct, un dreptunghi, un poligon sau un vector. \verb|QGraphicsScene::itemAt()| returnează elementul 
    de la cel mai înalt nivel de suprapunere într-un anumit punct. Toate funcțiile returnează elementele într-o ordine 
    descendenta în legătura cu eventuale suprapuneri.\newline

    \begin{lstlisting}[language=C++]
    QGraphicsScene scene;
    QRectF rectangle(0, 0, 100, 100);
    QGraphicsRectItem *rectItem = scene.addRect(rectangle);
    QGraphicsItem *item = scene.itemAt(50, 50);
    \end{lstlisting}

    Arhitectura de propagare a evenimentelor a clasei QGraphicsScene asigura transmiterea evenimentelor către 
    elemente dar și propagarea evenimentelor intre ele. Daca scene primește un eveniment legat de mouse la o anumita poziție, 
    scena transmite mai departe evenimentul către toate elementele care se afla în acel punct.\newline

    Scena totodată poate controla starea elementelor, de exemplu dacă elementele sunt selectate, sau sunt în focus relativ cu 
    evenimente de la tastatura. Mai multe elemente pot fi selectate folosind \verb|QGraphicsScene::setSelectionArea()|. 
    Pentru a accesa toate elementele selectate se poate folosi \verb|QGraphicsScene::selectedItems()|. O alta stare tratata este cea 
    de focus pe un anumit element. Se poate seta prin apelarea funcției \verb|QGraphicsScene::setFocusItem()| sau \verb|QGraphicsItem::setFocus()|, 
    iar pentru a accesa elementul curent în focus \verb|QGraphicsScene::focusItem()|.\newline

    \item View
    
    QGraphicsView furnizează un widget prin care se poate vizualiza conținutul unei scene. Se pot atasa mai multe view-uri la 
    aceeași scena dacă este nevoie. 

    \begin{lstlisting}[language=C++]
        QGraphicsScene scene;
        myPopulateScene(&scene);

        QGraphicsView view(&scene);
        view.show();

    \end{lstlisting}

    View-ul poate primii evenimente din exterior de la tastatura sau mouse și le trimite mai departe la scene 
    (făcând conversia la coordonatele scenei dacă este nevoie, deoarece în unele cazuri când scene este mai mare 
    decât fereastra view-ului, view-ul nu poate afișa toată scena, ci doar o porțiune).\newline
    
    Folosind matricea de transformare, QGraphicsView::transform(), view-ul poate scala sau rotii scena. 
    Pentru convenienta QGraphicsView oferă funcții pentru convertirea la, sau de la, coordonatele scenei la 
    coordonatele view-ului. \verb|QGraphicsView::mapToScene()| si \verb|QGraphicsView::mapFromScene()|.\newline

    \item Elementul
    
    QGraphicsItem este clasa de baza pentru elementele grafice dintr-o scena. Graphics View oferă câteva elemente grafice 
    standard pentru forme simple, precum dreptunghi  (\verb|QGraphicsRectItem|), elipsa (\verb|QGraphicsEllipseItem|), dar și text 
    (\verb|QGraphicsTextItem|). De obicei utilizatorul specializează clasa de baza prin moștenire pentru a obține elemente 
    grafice de forme diferite.\newline

    QGraphicsItem suporta următoarele functionalitati:
    \begin{itemize}
        \item evenimente de mouse (click, hover, scroll-wheel, etc.)
        \item evenimente de tastatura
        \item evenimente meniu de context
        \item grupare, fie prin relație părinte-copil, fie prin clasa QGraphicsItemGroup
        \item detecție de coliziuni
    \end{itemize}

    Toate elementele se afla într-un sistem de coordonate local, și la fel ca \verb|QGraphicsView|, conțin funcții pentru 
    convertirea la coordonatele scenei. Totodată la fel ca și \verb|QGraphicsView| sistemul de coordonate poate fi transformat 
    folosind o matrice, \verb|QGraphicsItem::transform()|. Acest lucru este folositor la scalarea sau rotirea individuala.\newline

    Elementele pot conține si alte element (copii). Transformările părintelui sunt moștenite de toți copiii.\newline
    
    \verb|QGraphicsItem| suporta și detecție de coliziuni, prin funcțiile \verb|QGraphicsItem::shape()| și \verb|QGraphicsItem::collidesWith()|, 
    ambele funcții virtuale care pot fi suprascrise.\newline
\end{itemize}

\subsection{Aplicare}

Majoritatea acestor noțiuni au fost folosite la dezvoltarea aplicație, interfața grafica a aplicatie avand la baza Qt. 
Fiecare tab deschis conține un \verb|QGraphicsView| cu o scena cu toate elementele. 
View-ul este dinamic iar utilizatorul poate interacționa cu el, prin zoom sau prin schimbarea întregii scene, 
selectând diferite perspective.\newline 

Fiecare element grafic din aplicație, moștenește clasa de baza QGraphicsItem, 
fiind specializat pentru diferitele tipuri care trebuie afișate. De exemplu obiectele de tip Motion body sunt 
afișate ca dreptunghiuri albastre de diferite forme în funcție de datele din fișierul sursa. Aceste elemente 
pot fi mutate cu mouse-ul. Elementele de tip Motion body pot fi conectate de alte elemente Motion body fie prin 
printr-un Joint sau Connector. La fel pentru Joint și Connector s-au folosit implementări separate pentru afișarea 
diferitelor tipuri.\newline 

Utilizatorul poate interacționa cu toate elementele prin click-dreapta pentru a deschide un meniu de context. 
Prin acest meniu el poate schimba culoarea, reseta culoarea sau deschide un panou în care sunt afișate informații adiționale 
elementului.\newline






